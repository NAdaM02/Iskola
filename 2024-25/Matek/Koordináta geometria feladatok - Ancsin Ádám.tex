\documentclass[a4paper,12pt]{article}
\usepackage[a4paper,left=25mm,right=25mm,top=25mm,bottom=25mm]{geometry}
\usepackage{t1enc}
\usepackage[utf8]{inputenc}
\usepackage[magyar]{babel}
\usepackage{tikz}
\usepackage[normalem]{ulem}
\usepackage{amsmath}

\begin{document}

\textbf{Koordináta geometria feladatok - Ancsin Ádám}
\\
\\ \underline{\textit{$\textbf{2. feladat} $}}
\\
\\ a)
\\
\\ Pontok:
\\ $A\left(\sqrt{7}, \quad 2\right)$ \qquad $B\left(-1, \quad 1+\sqrt{7}\right)$
\\
\\ A két pont távolsága függőleges és víszintes komponensekre bontható szét; \textit{itt} ezek:
\\
\\ > víszintes:\quad $d_{x} = \sqrt{7} + 1$
\\ > függőleges: $d_{y} = \left(1+ \sqrt{7}\right) - 2 = \sqrt{7} - 1$
\\
\\ Végül, az ezen távolságok által meghatározott derékszögű háromszög átfogójának hossza megegyezik a két pont távolságával:
\\
\\ $d = \sqrt{{d_x}^2 + {d_y}^2} = \sqrt{{\left(\sqrt{7} + 1\right)}^2 + {\left(\sqrt{7} - 1\right)}^2} = \sqrt{2\cdot\sqrt{7}^2 + 2\cdot1^2} = \sqrt{2\cdot7 + 2} = \sqrt{16} =\textbf{4}$
\\
\\ \uwave{\textit{Tehát a távolság $A$ és $B$ pont között: 4.}}
\\
\\
\\ b)
\\
\\ Általánosan (A és B pontok között):
\\
\\ > $d_x = b_x - a_x$
\\ \textit{Ennyit kell elmozdulni A-ból vízszintesen, hogy a B-vel azonos x pozícióval rendelkezzünk.}
\\
\\ > $d_y = b_y - a_y$
\\ \textit{Ennyit kell elmozdulni A-ból függőlegesen, hogy a B-vel azonos y pozícióval rendelkezzünk.}
\\
\\ Végül, az ezen távolságok által meghatározott derékszögű háromszög átfogójának hossza megegyezik a két pont távolságával:
\setlength{\abovedisplayskip}{0pt}
\setlength{\belowdisplayskip}{0pt}
\begin{center}
\begin{align*}
d &= \sqrt{{d_x}^2 + {d_y}^2} \\
d &= \sqrt{\left(b_x - a_x\right)^2 + \left(b_y - a_y\right)^2} \\
(d &= \sqrt{\left(a_x - b_x\right)^2 + \left(a_y - b_y\right)^2})
\end{align*}
\end{center}
\pagebreak
\underline{\textit{$\textbf{3. feladat} $}}
\\
\\ a) Egy olyan egyenes, amely átmegy az $A(2,3)$ és a $B(5,-1)$ pontokon.
\\
\\ \indent\textit{A lineáris egyenletek képlete:} $y = a \cdot x + c$
\\
\\ Az egyenes meredeksége a két pont alapján:
\\ $x \rightarrow x+3 \quad \Rightarrow \quad y \rightarrow y-4$
\\
\\ $\Rightarrow \quad \frac{dy}{dx} = -\frac{4}{3} =$ \textit{a háromszög meredeksége}
\\
\\ Így tudjuk, a keresett egyenes párhuzamos az $y = \frac{-4}{3} \cdot x$ egyenessel.
\\
\\ A konstans tagot (az eltolást) pedig a megadott pontok alapján számíthatjuk ki.
\\ \indent Tudjuk, hogy a függvény $x=2$ helyen $y=3$ értéket vesz fel, \textit{így}
\\ \indent $\rightarrow$
\\ \indent \indent ($y = a \cdot x+c$)
\\ \indent \indent $3 = \frac{-4}{3} \cdot 2 + c$
\\ \indent \indent $c = 3 - \frac{-8}{3}$
\\ \indent \indent $c = \frac{17}{3}$
\\
\\ A feladatnak megfelelő egyenes képlete tehát:
\\ $$y = \frac{-4}{3}x + \frac{17}{3}$$
\\
\\
\\ b) Egy félkör
\\
\\ A körök egyenletének képlete:\quad $\sqrt{x^2 + y^2} = r$
\\
\\ \indent Hogy elhagyjuk az $y$ negatív értékeit azt a tényt használhatjuk ki, hogy ha a $\sqrt{ }$-jel alatt negatív szám van, azt nem értelmezzük, így azon helyeken nem vesz fel a függvény értéket (a valós számok halmazában).
\\
\\ $\rightarrow$ Így az alábbi egyenleteket kaphatjuk:
\\ \indent ($\sqrt{x^2 + y^2}$) $\Rightarrow$ $\sqrt{x^2 + \left(\sqrt{y}\right)^4} = r$
\\ \indent \indent $\rightarrow$ Amennyiben ezt nem találjuk korrektnek -- mivel az egyenlet bizonyos tekintetben könnyen egyszerűsíthető a teljes kör képletére -- az alábbi egyenlethez fordulhatunk:
\\ $$\sqrt{\frac{|y|}{y}(x^2 + y^2)} = r$$
\\ \textit{Ez valóban egy $r$ sugarú félkört rajzol ki, ugyanis hasonló egy teljes kör képletéhez, míg amennyiben $y$ negatív, értékét nem értelmezzük.}
\\
\\ Mivel a feladat \textit{egy} félkörre utal, egy $3$ sugarú félkör képlete: $\sqrt{\frac{|y|}{y}(x^2 + y^2)} = 3$
\\
\\ c) Két különböző pont (és semmi más).
\\
\\ Egy egyenlet, melynek pontosan 2 (különböző) megoldása van.
\\
\\ Úgy szabályozhatunk le könnyen egy változót, hogy ha egy adott értékhez való eltérése nem $0$, akkor annak abszolút értékenek ellentetje negatív. Így annak gyöke csak egy adott érték esetén értelmes.
\\ $\sqrt{-|a-x|}$ (= 0)
\\
\\ Ezt könnyen módosíthatjuk úgy, hogy két érték esetén is megfeleljen a feltételnek.
\\ $\sqrt{-|a-|x||} (= 0)$ , mely a korábbi ``érvényes'' $x$ értékhez képest, már annak ellentetjével is értelmezhető.
\\
\\ Mivel 2 pontot szeretnénk létrehozni, mondhatjuk, hogy $x$ ``lehessen'' 2-féle, míg $y$, csak 1 értéket vehessen fel.
\\ Az így létrehozott függvény:
\\ \indent $\sqrt{-|a-|x||} \cdot \sqrt{-|b-y|} = 0$
\\ 
\\ \indent Ez pontosan a $P(a, b)$ és $Q(-a, b)$ pontokban értelmezhető.
\\
\\ \textit{Példa:} $\sqrt{-|3-|x||} \cdot \sqrt{-|2-y|} = 0$
\\ \indent Mely a $(3, 2)$ és $(-3, 2)$ pontokban értelmezhető.
\\
\\
\\
\underline{\textit{$\textbf{4. feladat} $}}
\\
\\ Amennyiben a $BC$ egyenes x-tengellyel bezárt szögét meg tudjuk határozni, a $C$ pont oordinátája is meghatározható. (Egy ábra segítségével könnyen belátható, hogy:)
\\
\\ $[$A $BC$ egyenes x-tengellyel bezárt szöge$] = 180^{\circ} + [AB$ x-tengellyel bezárt szöge$] - 60^{\circ}$
\\
\\ $[AB$ x-tengellyel bezárt szöge$] = \tan{}[AB$ meredeksége$]$
\\
\\ $[AB$ meredeksége$] = \frac{d_{AB,y}}{d_{AB,x}} = \frac{7 - 2}{-4 - 1} = \frac{5}{-5} = -1 $
\\
\\ $\Rightarrow [AB$ x-tengellyel bezárt szöge$] = -45^{\circ} $
\\
\\ \indent \indent $\Rightarrow [$A $BC$ egyenes x-tengellyel bezárt szöge$] = 180^{\circ} + (-45^{\circ}) - 60^{\circ} = \alpha = 75^{\circ} $ 
\\
\\ \textit{Ezt} és $C$ abszcisszáját felhasználva:
$$\tan(\alpha) = \frac{c_y - b_y}{c_x - b_x}$$
$$\tan(75^\circ) = \frac{p - (-4)}{11 - 7} = \frac{p + 4}{4}$$
$$p = 4(\tan{75^\circ}-1)$$
$$p = 4(2+\sqrt{3}-1)$$
$$p = 4 + \sqrt{48}$$


\end{document}